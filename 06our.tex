

\chapter{Opal toolset}

\todo[color=green]{implement proof-of-concept low-level tools to access available devices,}

In order to be able to study the self-encrypting drives, specifically the most popular Opal drives, at a closer level, we have implemented a toolset for controlling and inspecting Opal devices. 
We have focused on implementation that would not require any additional tooling or libraries and would allow as much control over the communication with the Opal device. This approach allows the project to be not only used on a system without introducing a plethora of additional packages but also for our code to be used in other projects without needlessly increasing the number of dependencies.
\todo{pridat --help do prilohy}

\section{Structure}

The code base is split into three levels. Each level implements a different part of the functionality that depends on the previous levels.

The lowest level is focused on the abstraction of the communication primitives. This includes primarily functions to build a method to be sent to the device and parse the received response, generic functions to start a session or set a value in a row of a table, and the implementation of IF-SEND and IF-RECV commands abstracting ATA and NVMe system calls.

% The middle level implements entire functions, like setup tper... 

The middle level joins the lower-level methods into complete functions. Each function represents one of the methods and provides a C interface to invoke the method. Internally the functions start a session with the device, build the method according to the function's arguments, send the method to the device, retrieve the response, and parse the response to find whether to command was successful and to potentially return the data to the caller.

Finally, the top level combines the two previous levels to create a usable front end usable through a command line.

\section{Provided utilities}

Currently, there are implemented three front ends: \verb|rng| to produce random numbers generated by the selected Opal disk, the \verb|control| to manipulate the status of the selected Opal device, and the \verb|discovery| to inspect the selected Opal device.

\subsection{Random number generation}


The \verb|rng| is the most straightforward front end of the introduced three. Its only goal is to produce random numbers generated by the disk quickly. However, all the disks we have tested implemented only the minimal Opal requirements, which guarantee a limit of only one active session, packets containing only one method, and the Random method providing only 32 bytes at once. Such properties do not leave much space for optimisation. The throughput this utility can achieve with the disks tested by us is between 500 bytes per second and 35000 bytes per second, depending on the disk. Although to obtain a sufficient amount of data, a considerable amount of time is required, this utility may be used to collect random numbers for testing the randomness of the disk's RNG.
The program repeatedly opens sessions, and during each session, it repeatedly sends the Random method. The total number of sessions and the number of bytes acquired during each session can be specified using the arguments as shown in Listing~\ref{fig:rng_output}.

\begin{lstlisting}[language=Bash,caption={Usage and output of the \texttt{rng} utility},label={fig:rng_output}]
# ./rng /dev/sdb 24 2
e993f67588190fc5d6d0ef7c7febf1944c566e211ba10352
29b0897bb4ddec1d71d89c95fc945a031574936c3c687477
\end{lstlisting}

\subsection{Device management}

The \verb|control| utility provides access to basic control over the device.
The utility serves as a wrapper around the middle-level functions described earlier.
As of right now, the utility provides access to the following commands:

\begin{itemize}
    \item \verb|psid_revert| command performs PSID revert using the Revert method as the PSID authority~\cite{tcg-psid}, resetting the device into the factory state.
    \item \verb|setup_tper| command takes ownership of the TPer by setting the password of the SID authority to the chosen value.\todo{(also enables programmatic reset, but that's to change, ... maybe)}
    \item \verb|setup_user| command enables Locking SP's user authority and sets its password.
    \item \verb|setup_range| command enables the locking range and sets up its ReadLocked and WriteLocked ACL and its other parameters, such as the range of the locking range, ReadLockEnabled, and WriteLockEnabled.
    \item \verb|unlock| command sets the ReadLocked and WriteLocked columns of a locking range, locking or unlocking the range for reading or writing.
    \item \verb|reset| command sends the TPER\_RESET command to the device~\cite{tcg-storage-core}.
\end{itemize}

An example of usage of the \verb|control| utility can be seen in Figure~\ref{fig:control_output}, where we present a basic use case scenario.

    \begin{lstlisting}[language=bash,caption={Usage of the \texttt{control} utility},label={fig:control_output},keywordstyle=\color{black}]
# Reset the device to the factory state.
./control psid_revert /dev/sda                 \
    --verify-pin ${PSID}
# Take ownership over the device.
./control setup_tper /dev/sda                  \
    --assign-pin 0000
# Create a user authority.
./control setup_user /dev/sda                  \ 
    --user 1 --verify-pin 0000                 \
    --assign-pin 1111
# Create a locking range.
./control setup_range /dev/sda                 \ 
    --locking-range 1                          \
    --locking-range-start 0                    \
    --locking-range-length 512                 \
    --user 1 --verify-pin 0000
# Lock reading of the locking range.
./control unlock /dev/sda                      \
    --user 1 --verify-pin 1111                 \ 
    --locking-range 1 --read-locked 1
\end{lstlisting}


\subsection{Disk properties discovery}
\label{utility_discovery}

The final front end, the \verb|discovery| utility, provides information about the Opal disk.
This utility provides, other than the Discovery information introduced in Section~\ref{section:opal_capability_discovery} also other useful information, such as the identify commands of the drive interfaces (Identify command for NVMe devices~\cite{nvme-express-base-specification} and IDENTIFY DEVICE command for ATA devices~\cite{acs-3}).
Level 0 Discovery and Level 1 Discovery information are acquired easily, as they are defined as they may be acquired using a simple IF-RECV command with specific ComID % and Protocol ID 
or using the Parameters method without any parameters, respectively.
The Level 2 Discovery, on the other hand, is more complicated.
\todo[color=green]{Level 2 is actually just tables 'AccessControl' and 'ACE'!!!!!, mention that we actually expand the meaning to cover all tablesf,,, just rerereread it. it says access to any table, but access control tables decide who can... i think}

% We would like to include not only the AccessControl and ACE table in this process but also other interesting tables, such as the TPerInfo table, containing information about the firmware version, or the SecretProtect table, containing information about the mechanism used to protect the DEKs. 
% Because of this, instead of the definition of Level 2 Discovery presented in section~\ref{section:opal_capability_discovery}, we will use a modified version.
% This modified version will try to include every table in the TPer.
For Level 2 Discovery, we would like to include as many tables as possible. The Opal standard defines the mandatory and optional tables, yet we would like to include also any table that is vendor specific.
We also have a few limiting factors that may limit the number of tables we will be able to read.
First of all, we do not want to change the state of the device. This includes actions such as reverting the device to the factory state or taking ownership of it. Even if the state is potentially reversible without data loss, the risk of the disk not working correctly surpasses the potential gain from more information. This might notably affect the collection of information from the Locking SP if it is not activated
% ...e.g. tables of locking SP might be unavailable if it was not activated (but the SP will still be mentioned in the SP table of Admin SP).
Another requirement is that no authorisation should be required. Instead, we will use the password-less Anybody authority. This again restricts the set of tables that are accessible. However, according to the definitions of access rights in the Opal standard~\cite{tcg-opal2}, this mainly affects secrets, such as passwords of Authorities or DEKs, and the current configuration of locking ranges, such as their start and length. Such information does not speak about the properties and capabilities of the device itself.

\subsection{Implementation of disk properties discovery}

The process we use in order to print the contents of every table is as follows:
Firstly we iterate using the Next method over the SP table of the Admin SP to get a list of existing SPs. As long as the Admin SP is not frozen or disabled, this should always succeed since the Admin SP is in the factory state by default, and its SP table is always accessible by the Anybody authority. 
With the list of SPs, we now may successively iterate using the Next method over the SPs and, for every one of them, get a list of their tables from the Table table, the same way we did with the SP table. The Table table is again always accessible by the Anybody authority.
We then go through the found tables and try to get a list of their rows using the Next method. If this succeeds, we can then use the Get method to get each row of the table. However, some of the tables do not implement the Next method. This means that if we want to get the information from the rows of such tables, we have to guess the UIDs of the rows. Since each table can have up to $2^{32}$ sparsely distributed rows, simply iterating over all possible values is out of the question. In those cases, we only guess that a row with the lowest possible UID might exist and try to read information from it.

Even after finishing the crawling, there might still be some tables left that we have not found. Even though the Table table should identify the tables of the SP, from our experience, it often did not contain all of them. To solve this issue, we have a list of tables which are described in the Opal standard~\cite{tcg-opal2} but were not found in the Table table of some of our tested disks. We then process these tables separately.

During the process, we print the acquired information. We use the JSON format with SPs, tables and rows as dictionaries. The rows then contain the columns as keys and the cells as values. Since the cells can contain diverse types, we have decided to 
parse only the outermost ones. This means that, for example, for lists, we print the inner bytes in hexadecimal form surrounded by square brackets.

% print it as a string if all values are ascii,
% print it surrounded by square brackets if list
% print it as bytes if the outermost is bytes
% otherwise just print the bytes



\begin{figure}
    \centering
\begin{lstlisting}[language=Python]
# Admin SP
 "0x0000020500000001": {
# Table table
  "0x0000000100000000": {
# Table table row
   "0x0000000100000001": {
# UID
    "0x00": "0x0000000100000001",
# Name
    "0x01": "(Table) 0x5461626c65",
# CommonName
    "0x02": "(Table) 0x5461626c65",
# TemplateID
    "0x03": "0x0000000000000000",
# Kind = Object
    "0x04": "0x01",
...
\end{lstlisting}
    \caption{Excerpt of \texttt{discovery} output from Kingston KC600. Comments were added for clarity.}
    \label{fig:discovery_example}
\end{figure}


An example of the output of this process can be seen in Figure~\ref{fig:discovery_example}. Depicted there is a part of the output of this utility containing the start of the Table table's first row. The nested structure contains the outermost key corresponding to the Admin SP, followed by a key for the Table table and a key for its first row. 
This first row holds metadata about the Table table, such as its name or the template that provides this table.

\todo{maybe use better example? where the common name is not used}

% , containing self-reference, can be seen. % and the Anybody ACE.






\todo{wtf, maybe just choose a different row....., are the comments I added better?, add some text that te comments are not part of the output though...}
\todo{reword so that the figure is inline... looks nicer :)}





%    "0x0000000800000001": {
%     "0x00": "0x0000000800000001",
%     "0x01": "(ACE_Anybody) 0x4143455f416e79626f6479",
%     "0x02": "(ACE_Anybody) 0x4143455f416e79626f6479",
%     "0x03": "[f2a400000c05a80000000900000001f3]",
%     "0x04": "[]"
%    },
% ...




% something about how we don't translate the types during printing ...

% iterate over each table to get either list of rows in that table or try to guess that the first row will contain information