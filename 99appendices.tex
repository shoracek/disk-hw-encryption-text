\appendix

\chapter{Patterns in disk content}
\label{appendix:rng_pattern}

When data is encrypted using symmetric cipher under CBC mode with one key and decrypted with another, the result should be random.
However, this is not the case with one of the disks we tested. Using this approach, SanDisk X300s contains data that may at first sight seem random, but after a deeper analysis, some patterns can be found.

In order to analyse the data present on the Opal disk after the DEK is regenerated, we used our utility \verb|repeat_finder.py|. This simple tool iterates a file and finds repeating chunks of data. In our case, each chunk was 2048 bytes long. Each repeating chunks is printed as list of its locations, divided by the chunk size. An example of part of this output can be seen in figure~\ref{fig:rng_pattern_fig}. In this concrete example, there are two patterns we can see.

\begin{lstlisting}[caption={Found patterns on zeroed disk},label={fig:rng_pattern_fig},language=Python]
[0, 128, 32432]
[1, 3, 5, 7, 9, 11, 13, 15]
[2, 130, 32434]
[4, 132, 32436]
[6, 134, 32438]
[8, 136, 32440]
[10, 138, 32442]
[12, 140, 32444]
[14, 142, 32446]
[16, 32672]
[17, 19, 21, 23, 25, 27, 29, 31, 33, 35, 37, 39, 41, 43, 45, 47, 49, 51, 53, 55]
[18, 32674]
    \end{lstlisting}
The first pattern can be seen on the first line, where the chunk repeats after 128 and 32432 chunks, and the pattern repeats several times for even locations. This pattern continues on throughout the entire analysed file, changing the value of the repeated bytes and the offsets and number of  repeats  once in a while.
The second pattern can be seen on the second line, where the chunk is located in every odd location. This pattern continues on throughout the entire analysed file, changing the value of the repeated bytes once in a while.

It should be noted that such patterns emerged only when the disk contents were overwritten with a repeating bytes, such as all zeroes. After repeating the writing of repeating bytes and regenerating of the DEK, the patterns changed. 

However, it is enough for two chunks to be the same. The listing \ref{fig:rng_pattern_fig_some} contains the patterns acquired from a disk that had its content randomized, except for two chunks. The bytes in fourth chunk were the same as the bytes in tenth chunk, an evidence of this can be seen in the listing.

\begin{lstlisting}[caption={Found patterns on disk with two same chunks},label={fig:rng_pattern_fig_some},language=Python]
[0]
[1]
[2]
[3, 9]
[4]
[5]
[6]
[7]
[8]
[10]
[11]
[12]
\end{lstlisting}


The source code of the utility and few of the outputs can be found in the attachments.