%%%%%%%%%%%%%%%%%%%%%%%%%%%%%%%%%%%%%%%%%%%%%%%%%%%%%%%%%%%%%%%%%%%%
%% I, the copyright holder of this work, release this work into the
%% public domain. This applies worldwide. In some countries this may
%% not be legally possible; if so: I grant anyone the right to use
%% this work for any purpose, without any conditions, unless such
%% conditions are required by law.
%%%%%%%%%%%%%%%%%%%%%%%%%%%%%%%%%%%%%%%%%%%%%%%%%%%%%%%%%%%%%%%%%%%%

\documentclass[
  digital, %% The `digital` option enables the default options for the
           %% digital version of a document. Replace with `printed`
           %% to enable the default options for the printed version
           %% of a document.
%%  color,   %% Uncomment these lines (by removing the %% at the
%%           %% beginning) to use color in the printed version of your
%%           %% document
  oneside, %% The `oneside` option enables one-sided typesetting,
           %% which is preferred if you are only going to submit a
           %% digital version of your thesis. Replace with `twoside`
           %% for double-sided typesetting if you are planning to
           %% also print your thesis. For double-sided typesetting,
           %% use at least 120 g/m² paper to prevent show-through.
  nolof,     %% The `lof` option prints the List of Figures. Replace
           %% with `nolof` to hide the List of Figures.
  nolot,     %% The `lot` option prints the List of Tables. Replace
           %% with `nolot` to hide the List of Tables.
]{fithesis4}
%% The following section sets up the locales used in the thesis.
\usepackage[resetfonts]{cmap} %% We need to load the T2A font encoding
\usepackage[T1,T2A]{fontenc}  %% to use the Cyrillic fonts with Russian texts.
\usepackage[
  main=english, %% By using `czech` or `slovak` as the main locale
                %% instead of `english`, you can typeset the thesis
                %% in either Czech or Slovak, respectively.
%   english, german, russian, czech, slovak %% The additional keys allow
]{babel}        %% foreign texts to be typeset as follows:
%%
%%   \begin{otherlanguage}{german}  ... \end{otherlanguage}
%%   \begin{otherlanguage}{russian} ... \end{otherlanguage}
%%   \begin{otherlanguage}{czech}   ... \end{otherlanguage}
%%   \begin{otherlanguage}{slovak}  ... \end{otherlanguage}
%%
%% For non-Latin scripts, it may be necessary to load additional
%% fonts:
\usepackage{paratype}
\def\textrussian#1{{\usefont{T2A}{PTSerif-TLF}{m}{rm}#1}}
%%
%% The following section sets up the metadata of the thesis.
\thesissetup{
    date        = \the\year/\the\month/\the\day,
    university  = mu,
    faculty     = fi,
    type        = mgr,
    department  = Department of Computer Systems and Communications,
    author      = Štěpán Horáček,
    gender      = m,
    advisor     = {Ing. Milan Brož, Ph.D.},
    title       = {Hardware-encrypted disks in Linux},
    TeXtitle    = {Hardware-encrypted disks in Linux},
    keywords    = {keyword1, keyword2, ...},
    TeXkeywords = {keyword1, keyword2, \ldots},
    abstract    = {%
      This is the abstract of my thesis, which can

      span multiple paragraphs.
    },
    % thanks      = {%
    %   These are the acknowledgements for my thesis, which can
    %   span multiple paragraphs.
    % },
    bib         = bibliography.bib,
    %% Remove the following line to use the JVS 2018 faculty logo.
    facultyLogo = fithesis-fi,
}
\usepackage{makeidx}      %% The `makeidx` package contains
\makeindex                %% helper commands for index typesetting.
\usepackage[acronym]{glossaries}          %% The `glossaries` package
\renewcommand*\glspostdescription{\hfill} %% contains helper commands
\loadglsentries{glossary.tex}  %% for typesetting glossaries
% \makenoidxglossaries                      %% and lists of abbreviations.
%% These additional packages are used within the document:
\usepackage{paralist} %% Compact list environments
\usepackage{amsmath}  %% Mathematics
\usepackage{amsthm}
\usepackage{amsfonts}
\usepackage{url}      %% Hyperlinks
\usepackage{markdown} %% Lightweight markup
\usepackage{listings} %% Source code highlighting
\lstset{
  basicstyle      = \ttfamily,
  identifierstyle = \color{black},
  keywordstyle    = \color{blue},
  keywordstyle    = {[2]\color{cyan}},
  keywordstyle    = {[3]\color{olive}},
  stringstyle     = \color{teal},
  commentstyle    = \itshape\color{magenta},
  breaklines      = true,
}
\usepackage{floatrow} %% Putting captions above tables
\floatsetup[table]{capposition=top}
\usepackage[babel]{csquotes} %% Context-sensitive quotation marks
\begin{document}

\chapter{Introduction}
% \markright{\textsc{Introduction}}
% \addcontentsline{toc}{chapter}{Introduction}

Just keeping the citations here, like~\cite{tcg-opal2}, \cite{linux-doc-inline}, \cite{sed-vulnerabilities}.

\section{Thesis description}


 The thesis aims to analyze existing approaches to using hardware-encrypted block devices (disks) in Linux and propose integrating such devices into the cryptsetup project.

The implementation should use the Linux kernel interface.

Student should
\begin{itemize}
    \item get familiar with and study available resources for self-encrypted drives, OPAL2 standard, block layer inline encryption,
    \item analyze and describe security of such drives,
    \item provide state-of-the-art overview of existing attacks,
    \item implement proof-of-concept extension to cryptsetup project (and possibly propose changes for Linux kernel).
\end{itemize}

The student should be familiar with C code for low-level system programming and cryptography concepts. 

\chapter{Hardware disk encryption}

\emph{Hardware disk encryption} is a technology that provides confidentiality of data stored on a storage device using encryption provided by hardware.

Some general overview about what it is, I guess., maybe talk about both sw and hw enc instead

Describe generic stuff: provisioning, locking, key types DEK, KEK, MEK, processes, locking ranges vs. FDE, ... keyslots?,,, (and for opal TPer, SP, ..)

mention LUKS somewhere...

The disk encryption process can be conducted in logically and physically different places, depending on the type. In the following sections of this chapter, we will further describe three such types.

\section{Self-encrypting drives}

at least try to mention ATA security,,, which means also mentioning opal,,,, and TCG Enterprise...

\section{Inline encryption hardware}

Compared to the previously described self-encrypting drives, inline encryption hardware is separated from the disk, and 

something about how it provides actually a way to check the encrypted content, right?

\section{Software encryption}

probably just a quick overview, ... maybe change the chapter to just "disk encryption"... or mention this just shortly at the start...

\subsection{Linux stack}

rethink where to put this... maybe tools?

\paragraph{dm-crypt}

\paragraph{fscrypt}

\chapter{fscrypt}

Assuming ext4 in this chapter...

\section{How to make it work?}

As of now, blk-layer inline encryption is supported only by two filesystems in Linux: ext4 and F2FS.
Need to use a kernel with \verb|CONFIG_FS_ENCRYPTION_INLINE_CRYPT| enabled.
Need to first mount the filesystem with the inline encryption flag:
\begin{lstlisting}[language=bash]
mount -t ext4 /dev/foo /mnt/foo -o inlinecrypt
\end{lstlisting}
It is not enough to specify the inline encryption flag, the encryption itself also must be enabled. Assuming ext4 filesystem, the encryption can be enabled after mounting like so:
\begin{lstlisting}[language=bash]
tune2fs -O encrypt "/dev/loop0"
\end{lstlisting}
After this, \verb|fscrypt| can be used as normal and it will use inline encryption for this filesystem. 

In order to encrypt a folder using a \verb|fscrypt|, the following must be done: an encryption key must be added and the encryption policy must be created.

\begin{lstlisting}[language=c]
int fd = open(pathname, O_RDONLY | O_CLOEXEC);

struct fscrypt_add_key_arg *key_request = calloc(1, sizeof(struct fscrypt_add_key_arg) + key_len);
struct fscrypt_policy_v2 policy_request = { 0 };

// add a key
key_request->key_spec.type = FSCRYPT_KEY_SPEC_TYPE_IDENTIFIER;
key_request->key_id = 0;
key_request->raw_size = key_len;
memcpy(key_request->raw, key, key_len);
ioctl(fd, FS_IOC_ADD_ENCRYPTION_KEY, key_request);

// set a policy
policy_request.version = 2;
policy_request.contents_encryption_mode = FSCRYPT_MODE_AES_256_XTS;
policy_request.filenames_encryption_mode = FSCRYPT_MODE_AES_256_CTS;
policy_request.flags = FSCRYPT_POLICY_FLAGS_PAD_8 | FSCRYPT_POLICY_FLAGS_PAD_16 | FSCRYPT_POLICY_FLAGS_PAD_32;
memcpy(policy_request.master_key_identifier, key_request->key_spec.u.identifier, FSCRYPT_KEY_IDENTIFIER_SIZE);
ioctl(fd, FS_IOC_SET_ENCRYPTION_POLICY, &policy_request);
\end{lstlisting}

This code sets up an encryption policy for the file specified by the pathname. 

\section{How does it work?}

\subsection{Setup}

During mounting the ``inlinecrypt''/\verb|SB_INLINECRYPT|  flag is written into the \verb|super_block| structure.

It all starts in \verb|__ext4_new_inode|. This is the internal function used when creating new inodes, called by functions such as \verb|ext4_create| when creating a new file.

The function (if it is not inode used for large extended attributes?) calls \verb|fscrypt_prepare_new_inode|.

\verb|fscrypt_prepare_new_inode -> fscrypt_setup_encryption_info ->| \verb|setup_file_encryption_key ->  fscrypt_select_encryption_impl|

In \verb|fscrypt_select_encryption_impl| there is actually the only place where the \verb|SB_INLINECRYPT| flag is used.
... Calls \verb|blk_crypto_config_supported| to check the device's crypto profile.
Afterwards, \verb|fscrypt_select_encryption_impl| function sets the \verb|(fscrypt_info *)ci->ci_inlinecrypt|.

\verb|setup_per_mode_enc_key| then sets the \verb|(fscrypt_info *)ci->ci_enc_key|.

\subsection{Usage}

The bio function is stored in \\ \verb|(struct bio *)bio->(struct bio_crypt_ctx *)bi_crypt_context|

Function \verb|fscrypt_set_bio_crypt_ctx| changes the file's bio to use inline encryption... simply calls the blk layer \verb|bio_crypt_set_ctx|.

Calling \verb|submit_bio| like normally ... \verb|__submit_bio| calls \verb|__blk_crypto_bio_prep|...

\blockquote{If the bio crypt context provided for the bio is supported by the underlying device's inline encryption hardware, do nothing.}

\verb|__blk_crypto_rq_bio_prep| however sets the context of the request to the one of the bio... 
After the bio prep  \verb|blk_mq_submit_bio| gets called (which calls \verb|blk_mq_bio_to_request|, and after that also \verb|blk_crypto_init_request->blk_crypto_get_keyslot| which updates the devices keyslot to contain the new key..., but does nothing if the device does not have keyslots)..... the info about the key to use then has to be acquired by the driver from \verb|request->|


Most important are probably structures \verb|blk_crypto_ll_ops| and \verb|blk_crypto_profile|... just two operations, program key and evict key. 



how to get crypto profile from outside..




Where does the hardware come to play?

\verb|ufshcd_exec_raw_upiu_cmd()->ufshcd_issue_devman_upiu_cmd()->ufshcd_prepare_req_desc_hdr()->ufshcd_prepare_req_desc_hdr_crypto()| sets the header with the correct keyslot.

\chapter{TCG Opal 2.0}

TCG Opal SSC (Security Subsystem Class) 2.0 is a specification for storage devices, aiming to provide confidentiality of stored data while the disk implementing this specification is turned off~\cite{tcg-opal2}. The Opal standard is one of the representants of the self-encrypting drive approach to hardware disk encryption.

developed by TCG (Trusted Computing Group)

maybe move under the SED chapter, but it is probably going to be too big for that..., and important


structure of the standards described in \url{https://trustedcomputinggroup.org/wp-content/uploads/TCGandNVMe_Joint_White_Paper-TCG_Storage_Opal_and_NVMe_FINAL.pdf}.



\section{Technical stuff}

I expect stuff here like important headers, codes, ...

SP = service provider, tables and methods that operate upon them

\subsection{Capability discovery}

begin with Level 0 discovery: \parencite[3.3.6]{tcg-storage-core} specifies the general security receives command for all TCG storage devices \cite[3.1.1]{tcg-opal2} specifies the Opal 2 specific parts of the header.

The Opal 2 feature header contains several important information usable in the future usage of the device. Most importantly the base ComID~\cite[3.3.2]{tcg-storage-core}. 

ComID are divided into static and dynamic (managed by ComID management).

\subsection{sessions}

In order to facilitate communication between the host and SP, sessions must be used. During these sessions, methods are used...

example of method: 

\begin{lstlisting}
SMUID.StartSession [
HostSessionID : uinteger,
SPID : uidref {SPObjectUID},
Write : boolean,
HostChallenge = bytes,
HostExchangeAuthority = uidref {AuthorityObjectUID},
HostExchangeCert = bytes,
HostSigningAuthority = uidref {AuthorityObjectUID},
HostSigningCert = bytes,
SessionTimeout = uinteger,
TransTimeout = uinteger,
InitialCredit = uinteger,
SignedHash = bytes ]
=>
SMUID.SyncSession [ see SyncSession definition in 5.2.3.2]
\end{lstlisting}

Each method call is defined by the caller UID, method UID, and values of its parameters.
The method calls are coded using tokens... start list, end list, optional argument start, ... short int, long int,....

To call a method use IF-SEND command, to get result of the method call use IF-RECV command.

In order to authentize the user, one can use many different ...



Methods are sent using packets. There are three types of packets: ComPackets, packets and subpackets. A single ComPacket can contain multiple packets and single packet can contain multiple subpackets. Each ComPacket is associated with a single ComID, each packet is associated with a single session.



\subsection{setup}
\subsection{set locking range}

global locking range is a special locking range that protects LBAs not protected by any of the other locking ranges.

\subsection{access...}
\subsection{...}


\section{Comparison with OPAL 1.0}

OPAL 2 disks are \emph{optionally} backwards compatible.

cite \url{https://trustedcomputinggroup.org/wp-content/uploads/TCG_Storage-Opal_SSC_FAQ.pdf}

\section{Comparision with other TCG's SSC}

Enterprise, Opalite, Pyrite, Ruby

\chapter{Existing tools}

\section{Cryptsetup}

... tool for disk encryption setup.

nothing for opal, right?

\section{sedutil}

supports opal, but the code seems to be abandoned

\chapter{Security of hardware encryption}

specify threat model, ,....,,, offline, online,,, offline multiple times???s

.. maybe change position of this chapter, since it's splitting tools and change to a tool

\section{Attacks on hardware encryption}

\chapter{OPAL extension for cryptsetup}

\chapter{Conclusion}

\chapter{Notes}

Write/read on locked range returns "Input/output error".

tested disks:
\begin{itemize}
    \item SanDisk SD7UB2Q5
    \item Samsung SSD 850
    \begin{itemize}
        \item problems with user authentication
    \end{itemize}
\end{itemize}

ATA drives need \verb|libata.allow_tpm| set to 1,,, but why?

templates --- base, locking,, -> SPs. Service providers of the trusted peripherals are issued based on templates.

Global locking range is a special locking range that controls any area of the disk that is not controlled by any other locking range.

Shadow MBR allows FDE disk pre-boot. Presents ``fake'' MBR that can unlock the disk, on successful unlocking hands the control over to the real MBR.

key types (Credential Table Group): \begin{itemize}
    \item \verb|C_PIN| --- password
    \item \verb|C_RSA_*| --- signing arbitrary input???, session startup
    \item \verb|C_EC_*| --- ec-mqv, ec-dh session startup
    \item \verb|C_AES_*|
    \item \verb|C_HMAC_*|
\end{itemize}

5.3.4.1.3

authority operations: 
\begin{itemize}
    \item None --- does not authenticate
    \item Password --- authenticated using a pin
    \item Sign --- challenge and response
    \item Exchange
    \item SymK
    \item HMAC
    \item TPerSign
    \item TPerExchange
\end{itemize}

Sessions may use secure messaging: 
use additionaly a start/synctrustedsession methods for challange-response/key exchange

CVE-2018-12037  --- ``An issue was discovered on Samsung 840 EVO and 850 EVO devices (only in "ATA high" mode, not vulnerable in "TCG" or "ATA max" mode), \dots'' --- wtf is ata high/ata max mode???

CVE-2018-12038


s \\
e \\
p \\
a \\
r \\
a \\
t \\
o \\
r \\



All that follows is from the core standard\cite{tcg-storage-core}.



\hline

\subsection{Trusted Peripheral}

Trusted peripheral (TPer) is a device located on the disk that procures the security of the data on disk.

\subsection{Security Provider}

A TPer consists of multiple Security providers (SP). SPs are defined as a set of tables, methods and an access control. Each SP is derived from a set of templates. These templates define ...

\hline

Security Provider (SP) --- tables, methods, access control

secure messaging --- integrity/authentication and/or confidentiality,,,,


single user mode --- feature set that ``locks'' the admin out --- admin can do only destructive actions upon the locking range, and only the user can actually unlock it



\hline

\subsection{Other subsystems/Structure of the standard}

The Opal standard is defined as a subsystem extending the TCG Storage Core standard. The TCG Storage Core standard specifies the core features and proprieties shared among several different types of subsystems, that extend the core functionality by specifying additional or mandatory features, each subsystem focused on a different use case. These subsystems are namely: \begin{itemize}
    \item Opal --- targeted at the corporate usage. Described more closely in the rest of the thesis.
    \item Opalite --- simplified Opal. Removes features such as locking ranges, decreases number of mandatory admins, users.
    \item Pyrite --- similar to Opalite, however does not require encryption, and instead offers only logical access control.
    \item Ruby --- subsystem focused on data centers and server drives. Offers only global range encryption, weaker configuration of access control, no pre-boot authentication support. Replacing older Enterprise subsystem. 
\end{itemize}

\hline

Trusted Computing Group (TCG), “Storage Work Group Use Case White Paper – v 1.0” \\
opal --- ``for corporate''
opalite --- simplified/worse opal, only global range, less admins and uesrs \\
pyrite --- non-encrypted opalite, only logical access control \\
enterprise --- only global range, no psid, no configurable access control, no pre-boot authentication, for scsi data centers devices, server drives
new - ruby --- replacing enterprise... enterprise + datacenter

``Data confidentiality and access control over TPer features and capabilities: \dots The protection provided by this
exclusive access extends to confidentiality of instructions and data in transit between the
trusted host application (or a TPM it uses) and the TPer''


secure messaging

sessions are using readers-writers lock.

SPs are combinations of templates, must contain base template (). other templates admin, clock, crypto, locking, log

``All SPs incorporate at least a subset of the Base Template’s tables and methods.'' -> is $\emptyset$ enough???

issuance = new SP from templates, personalization = customization of newly created SP (but can happen anytime...), initial data, authorities, 

SSC = ... ``A TPer MAY have only some of the capabilities (tables, methods, access controls, etc.) defined in this
Core Specification and MAY include additional capabilities through table definitions and/or methods. A
Security Subsystem Class SHALL NOT replace a capability called out in the Core Specification with the
same capability implemented in different tables, methods, and access controls.''

stream encoding - methods and tokens...,TLV...

tables -- bytes tables -- raw data, just bytes (rows addressed by row number); object tables (rows addressed by uid, SP-unique, non-reusable, for anti-spoofing)

object = row of table

SP templates -- base, admin, clock, crypto, locking, log,


interface -- protocol-independent, IF-SEND, IF-RECV commands 

\hline

\section{Communication}

ComID is a number identifying the caller (e.g. application of the host). ComID ensures that responses will be received by the correct application, since there can be at one time multiple callers.


\hline

COMID

ComID identifies the caller, each application has different ComID and therefore can communicate simultaneously, dynamic, static. Not a session, multiple sessions may use single comid. Then there are some states and transitions, active, associated, issued... Extended ComID for reusing ComID so that we can see if it is the old ComID or the new one (using extra two bytes).

PROTOCOL LAYERS

protocol layers --- session, management, communication, tper, interface, transport
(... the standard seems to imply that the comid is always managed??)

DISCOVERY

discovery levels : 3.3.5 : level 0 , level 1 Properties method of TPer, level 2 of SP

SESSIONS

sessions: regular and control (not going to care about control much, just between TPer session manager and host session manager); read/read-write mutexes..., 

session manager methods - properties, start/sync session, start/sync trusted session, close session

*trustedsession - used with PuK, SymK, and HMAC authorities, secure messaging

authorities -- used during session startup,,, host exchange authority -- exchange of session keys, implicit authentication;;; host signing authority -- challenge response authentication/startup method integrity, \verb|C_PIN| (password), , ,,, same for host->SP

METHODS

TRANSACTIONS 

\hline

\subsection{Transactions}

In order to facilitate safe execution of sequences of methods, the standard specifies transactions. Similarly to transactions in database systems, this feature enables one to revert effects of sequence of methods. This is done automatically in case the transaction is not finished, or if the transaction is manually aborted. However, not all the effects of methods are rolled back, such as logs. Nested transactions are supported, in which case the transaction is commited when the outermost transaction is finished.

\hline

--- like in database, commiting , etc.... optional (or implicit transaction on method level...), nested transaction commited when outermost finished, possible exceptions to rollbacks e.g. logs

Stream Flow Control -- interface (handles  sending IF-SEND/RECV commands across the interface ) and stream data (handles not overwhelming host and TPer) types;;; uses Credit Control Subpacket to signify the opposite device that it is ready to receive data (and how much)

Session Reliability --- ACKs, NACKs (SeqNumbers), timeouts,,, all optional

Synchronous Interface Communications --- alternative to the previous asynchronous communication... to make it more simple,,,, so IF-SEND to make TPer do something, IF-RECV to get the result, repeat ad nauseam


SP OPERATIONS

Special SP - admin SP -- maintains info about TPer and other SPs

SP -- Cartesian product of template (defining tables and methods, pretty sure I wrote about it earlier, but it is mentioned in the standard again) subsets, 

ACCESS CONTROL

\hline

\subsection{Access Control}

In order to provide access to methods only to authorized actors, the standard also defines access control. The access control provided a way to allow access to methods of the SP only after a knowledge of a secret has been proved. Information required to verify knowledge of the secret is called a credential and is stored in one of the corresponding SP's credential table. Each SP may contain several credential tables, one for each type of credential, such as pin, RSA key pair, or AES key.

In order to support authentication of more than one user or require multiple users at once access control rules are defined using Access Control Lists (ACL) containing Access Control Elements (ACE). Each entry in an ACL has an owner \dots. ACEs are Boolean expressions with inputs being authentication of authorities.

Other than an explicit authentication, an implicit authentication is also possible. In this authentication, the knowledge of secret is shown by successfully using encrypted communication channel.

\hline

invoker may have to know secret (secret + public part = Credentials, operation of proving knowledge = Authentication Operation, proving knowledge = Authentication )

explicit authentication - e.g. password validation, challenge response,,,
implicit authentication - e.g session key exchange

authority --- object in authority table, type is individual or class (afaik just a bunch of individual authorities,, so that they may be easily changed at once, OR of all the individuals)

Access control -- access control list consists of access control elements (boolean combination of authorities),,,
each SP has AdminExch authority,,, 
ACEs seem to be able to have different "owners",,,

\hline

\subsection{SP Issuance}

SP issuance is the process of creation of a new SP from a base template and a set of other templates. After an SP is issued, it can be personalized. Personalization of an SP is the process of 

\hline

ISSUANCE

issuance -- creation of SP from template, from admin SP, ,,,
personalization --- after issuance, ussing the adminexch authority of the new sp, fill in tables, set access control... etc

LIFE CYCLE

... SP can be disabled (still can authenticate, deleteSP and set get on SPinfo (to enable)), frozen (any attempt to open session fails)


s \\
e \\
p \\
a \\
r \\
a \\
t \\
o \\
r \\

... And that§s the introduction... what follows now is the method and table overview...


.....

\hline

\subsection{Templates}

The standard specifies several templates that can be used to create SPs.

\hline

Base Template --

Crypto Template ---- operates on Credential tables (the previously mentioned \verb|C_*|, tables of other SPs,,,), can do enc, dec, sig, ver, hash, hmac, xor

Locking Template ------ provides access control



\printbibliography[heading=bibintoc] %% Print the bibliography.

\end{document}
