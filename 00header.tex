%%%%%%%%%%%%%%%%%%%%%%%%%%%%%%%%%%%%%%%%%%%%%%%%%%%%%%%%%%%%%%%%%%%%
%% I, the copyright holder of this work, release this work into the
%% public domain. This applies worldwide. In some countries this may
%% not be legally possible; if so: I grant anyone the right to use
%% this work for any purpose, without any conditions, unless such
%% conditions are required by law.
%%%%%%%%%%%%%%%%%%%%%%%%%%%%%%%%%%%%%%%%%%%%%%%%%%%%%%%%%%%%%%%%%%%%

\documentclass[
  digital, %% The `digital` option enables the default options for the
           %% digital version of a document. Replace with `printed`
           %% to enable the default options for the printed version
           %% of a document.
%%  color,   %% Uncomment these lines (by removing the %% at the
%%           %% beginning) to use color in the printed version of your
%%           %% document
  oneside, %% The `oneside` option enables one-sided typesetting,
           %% which is preferred if you are only going to submit a
           %% digital version of your thesis. Replace with `twoside`
           %% for double-sided typesetting if you are planning to
           %% also print your thesis. For double-sided typesetting,
           %% use at least 120 g/m² paper to prevent show-through.
  nolof,     %% The `lof` option prints the List of Figures. Replace
           %% with `nolof` to hide the List of Figures.
  nolot,     %% The `lot` option prints the List of Tables. Replace
           %% with `nolot` to hide the List of Tables.
]{fithesis4}
%% The following section sets up the locales used in the thesis.
\usepackage[resetfonts]{cmap} %% We need to load the T2A font encoding
\usepackage[T1,T2A]{fontenc}  %% to use the Cyrillic fonts with Russian texts.
\usepackage[
  main=english, %% By using `czech` or `slovak` as the main locale
                %% instead of `english`, you can typeset the thesis
                %% in either Czech or Slovak, respectively.
%   english, german, russian, czech, slovak %% The additional keys allow
]{babel}        %% foreign texts to be typeset as follows:
%%
%%   \begin{otherlanguage}{german}  ... \end{otherlanguage}
%%   \begin{otherlanguage}{russian} ... \end{otherlanguage}
%%   \begin{otherlanguage}{czech}   ... \end{otherlanguage}
%%   \begin{otherlanguage}{slovak}  ... \end{otherlanguage}
%%
%% For non-Latin scripts, it may be necessary to load additional
%% fonts:
\usepackage{paratype}
\def\textrussian#1{{\usefont{T2A}{PTSerif-TLF}{m}{rm}#1}}
%%
%% The following section sets up the metadata of the thesis.
\thesissetup{
    date        = \the\year/\the\month/\the\day,
    university  = mu,
    faculty     = fi,
    type        = mgr,
    department  = Department of Computer Systems and Communications,
    author      = Štěpán Horáček,
    gender      = m,
    advisor     = {Ing. Milan Brož, Ph.D.},
    title       = {Hardware-encrypted disks in Linux},
    TeXtitle    = {Hardware-encrypted disks in Linux},
    keywords    = {disk encryption, self-encrypting disk, ...},
    % TeXkeywords = {keyword1, keyword2, \ldots},
    abstract    = {%
The thesis aims to analyze existing approaches to using hardware-encrypted block devices (disks) in Linux.

The implementation part should provide basic low-level tools for tools configuration and status check of such devices. Tools should use generic Linux kernel interfaces (like ioctl calls).

Student should \begin{itemize}
    \item get familiar with and study available resources for self-encrypted drives, OPAL2 standard, block layer inline encryption,
    \item analyze and describe security of such drives,
    \item provide state-of-the-art overview of existing attacks,
    \item implement proof-of-concept low-level tools to access available devices,
    \item evaluate and discuss the result.
\end{itemize}

The student should be familiar with C code for low-level system programming and cryptography concepts.

TODO TODOTODO TODOTODO TODOTODO TODOTODO TODOTODO TODOTODO TODOTODO TODOTODO TODOTODO TODOTODO TODOTODO TODOTODO TODOTODO TODOTODO TODOTODO TODOTODO TODOTODO TODOTODO TODOTODO TODO\todo{change me }

    },
    % thanks      = {%
    %   These are the acknowledgements for my thesis, which can
    %   span multiple paragraphs.
    % },
    bib         = bibliography.bib,
    %% Remove the following line to use the JVS 2018 faculty logo.
    facultyLogo = fithesis-fi,
}
\usepackage{makeidx}      %% The `makeidx` package contains
\makeindex                %% helper commands for index typesetting.
\usepackage[acronym]{glossaries}          %% The `glossaries` package
\renewcommand*\glspostdescription{\hfill} %% contains helper commands
\loadglsentries{glossary.tex}  %% for typesetting glossaries
% \makenoidxglossaries                      %% and lists of abbreviations.
%% These additional packages are used within the document:
\usepackage{paralist} %% Compact list environments
\usepackage{amsmath}  %% Mathematics
\usepackage{amsthm}
\usepackage{amsfonts}
\usepackage{url}      %% Hyperlinks
\usepackage{markdown} %% Lightweight markup
\usepackage{listings} %% Source code highlighting
\lstset{
  basicstyle      = \ttfamily,
  identifierstyle = \color{black},
  keywordstyle    = \color{blue},
  keywordstyle    = {[2]\color{cyan}},
  keywordstyle    = {[3]\color{olive}},
  stringstyle     = \color{teal},
  commentstyle    = \itshape\color{magenta},
  breaklines      = true,
}
\usepackage{floatrow} %% Putting captions above tables
\floatsetup[table]{capposition=top}
\usepackage[babel]{csquotes} %% Context-sensitive quotation marks


\usepackage{csvsimple}

\usepackage{listings}

\usepackage{tikz}

\usepackage{hyperref}

\usepackage{bytefield}
\usetikzlibrary{positioning}


\usepackage[textwidth=8em,disable]{todonotes}
% ,textsize=small,backgroundcolor=white,disable