
\chapter{Notes}

Write/read on locked range returns "Input/output error".

tested disks:
\begin{itemize}
    \item SanDisk SD7UB2Q5
    \item Samsung SSD 850
    \begin{itemize}
        \item problems with user authentication
    \end{itemize}
\end{itemize}

ATA drives need \verb|libata.allow_tpm| set to 1,,, but why?,,,, it's not just tpm, it's tcg commands in general e..gg \verb|trusted_send| receivew,,, it's encrypted "backdoor" ... https://git.kernel.org/pub/scm/linux/kernel/git/torvalds/linux.git/tree/drivers/ata/libata-scsi.c?id=HEAD

templates --- base, locking,, -> SPs. Service providers of the trusted peripherals are issued based on templates.

Global locking range is a special locking range that controls any area of the disk that is not controlled by any other locking range.

Shadow MBR allows FDE disk pre-boot. Presents ``fake'' MBR that can unlock the disk, on successful unlocking hands the control over to the real MBR.

key types (Credential Table Group): \begin{itemize}
    \item \verb|C_PIN| --- password
    \item \verb|C_RSA_*| --- signing arbitrary input???, session startup
    \item \verb|C_EC_*| --- ec-mqv, ec-dh session startup
    \item \verb|C_AES_*|
    \item \verb|C_HMAC_*|
\end{itemize}

5.3.4.1.3

authority operations: 
\begin{itemize}
    \item None --- does not authenticate
    \item Password --- authenticated using a pin
    \item Sign --- challenge and response
    \item Exchange
    \item SymK
    \item HMAC
    \item TPerSign
    \item TPerExchange
\end{itemize}

Sessions may use secure messaging: 
use additionaly a start/synctrustedsession methods for challange-response/key exchange

CVE-2018-12037  --- ``An issue was discovered on Samsung 840 EVO and 850 EVO devices (only in "ATA high" mode, not vulnerable in "TCG" or "ATA max" mode), \dots'' --- wtf is ata high/ata max mode???

CVE-2018-12038


s \\
e \\
p \\
a \\
r \\
a \\
t \\
o \\
r \\


\newpage

All that follows is from the core standard~\cite{tcg-storage-core}.




% Security Provider (SP) --- tables, methods, access control

secure messaging --- integrity/authentication and/or confidentiality,,,,




% Trusted Computing Group (TCG), “Storage Work Group Use Case White Paper – v 1.0” \\
% opal --- ``for corporate''
% opalite --- simplified/worse opal, only global range, less admins and uesrs \\
% pyrite --- non-encrypted opalite, only logical access control \\
% enterprise --- only global range, no psid, no configurable access control, no pre-boot authentication, for scsi data centers devices, server drives
% new - ruby --- replacing enterprise... enterprise + datacenter

``Data confidentiality and access control over TPer features and capabilities: \dots The protection provided by this
exclusive access extends to confidentiality of instructions and data in transit between the
trusted host application (or a TPM it uses) and the TPer''


secure messaging

sessions are using readers-writers lock.

SPs are combinations of templates, must contain base template (). other templates admin, clock, crypto, locking, log

``All SPs incorporate at least a subset of the Base Template’s tables and methods.'' -> is $\emptyset$ enough???

issuance = new SP from templates, personalization = customization of newly created SP (but can happen anytime...), initial data, authorities, 

SSC = ... ``A TPer MAY have only some of the capabilities (tables, methods, access controls, etc.) defined in this
Core Specification and MAY include additional capabilities through table definitions and/or methods. A
Security Subsystem Class SHALL NOT replace a capability called out in the Core Specification with the
same capability implemented in different tables, methods, and access controls.''

stream encoding - methods and tokens...,TLV...

tables -- bytes tables -- raw data, just bytes (rows addressed by row number); object tables (rows addressed by uid, SP-unique, non-reusable, for anti-spoofing)

object = row of table

SP templates -- base, admin, clock, crypto, locking, log,


interface -- protocol-independent, IF-SEND, IF-RECV commands 


COMID

ComID identifies the caller, each application has different ComID and therefore can communicate simultaneously, dynamic, static. Not a session, multiple sessions may use single comid. Then there are some states and transitions, active, associated, issued... Extended ComID for reusing ComID so that we can see if it is the old ComID or the new one (using extra two bytes).

PROTOCOL LAYERS

protocol layers --- session, management, communication, tper, interface, transport  
(... the standard seems to imply that the comid is always managed??)

DISCOVERY

discovery levels : 3.3.5 : level 0 , level 1 Properties method of TPer, level 2 of SP

SESSIONS

sessions: regular and control (not going to care about control much, just between TPer session manager and host session manager); read/read-write mutexes..., 

session manager methods - properties, start/sync session, start/sync trusted session, close session

*trustedsession - used with PuK, SymK, and HMAC authorities, secure messaging

authorities -- used during session startup,,, host exchange authority -- exchange of session keys, implicit authentication;;; host signing authority -- challenge response authentication/startup method integrity, \verb|C_PIN| (password), , ,,, same for host->SP

METHODS

TRANSACTIONS 


--- like in database, commiting , etc.... optional (or implicit transaction on method level...), nested transaction commited when outermost finished, possible exceptions to rollbacks e.g. logs

Stream Flow Control -- interface (handles  sending IF-SEND/RECV commands across the interface ) and stream data (handles not overwhelming host and TPer) types;;; uses Credit Control Subpacket to signify the opposite device that it is ready to receive data (and how much)

Session Reliability --- ACKs, NACKs (SeqNumbers), timeouts,,, all optional

Synchronous Interface Communications --- alternative to the previous asynchronous communication... to make it more simple,,,, so IF-SEND to make TPer do something, IF-RECV to get the result, repeat ad nauseam


SP OPERATIONS

Special SP - admin SP -- maintains info about TPer and other SPs

SP -- Cartesian product of template (defining tables and methods, pretty sure I wrote about it earlier, but it is mentioned in the standard again) subsets, 

ACCESS CONTROL

invoker may have to know secret (secret + public part = Credentials, operation of proving knowledge = Authentication Operation, proving knowledge = Authentication )

explicit authentication - e.g. password validation, challenge response,,,
implicit authentication - e.g session key exchange




s \\
e \\
p \\
a \\
r \\
a \\
t \\
o \\
r \\

... And that§s the introduction... what follows now is the method and table overview...


.....

\subsection{Templates}

The standard specifies several templates that can be used to create SPs.


Base Template --

Crypto Template ---- operates on Credential tables (the previously mentioned \verb|C_*|, tables of other SPs,,,), can do enc, dec, sig, ver, hash, hmac, xor

Locking Template ------ provides access control



TODO: differentiate between list output and number output 


% \chapter{TCG Opal 2.0}

% TCG Opal SSC (Security Subsystem Class) 2.0 is a specification for storage devices, aiming to provide confidentiality of stored data while the disk implementing this specification is turned off~\cite{tcg-opal2}. The Opal standard is one of the representants of the self-encrypting drive approach to hardware disk encryption.

% developed by TCG (Trusted Computing Group)

% maybe move under the SED chapter, but it is probably going to be too big for that..., and important


% structure of the standards described in \url{https://trustedcomputinggroup.org/wp-content/uploads/TCGandNVMe_Joint_White_Paper-TCG_Storage_Opal_and_NVMe_FINAL.pdf}.



% \section{Technical stuff}

% I expect stuff here like important headers, codes, ...

% SP = service provider, tables and methods that operate upon them

% \subsection{Capability discovery}

% begin with Level 0 discovery: \parencite[3.3.6]{tcg-storage-core} specifies the general security receives command for all TCG storage devices~\cite[3.1.1]{tcg-opal2} specifies the Opal 2 specific parts of the header.

% The Opal 2 feature header contains several important information usable in the future usage of the device. Most importantly the base ComID~\cite[3.3.2]{tcg-storage-core}. The base ComID, together with another parameter the number of ComIDs, define the range of possible static ComIDs.

% ComID are divided into static and dynamic (managed by ComID management).

% \subsection{sessions}

% In order to facilitate communication between the host and SP, sessions must be used. During these sessions, methods are used...

% example of method: 

% \begin{lstlisting}
% SMUID.StartSession [
% HostSessionID : uinteger,
% SPID : uidref {SPObjectUID},
% Write : boolean,
% HostChallenge = bytes,
% HostExchangeAuthority = uidref {AuthorityObjectUID},
% HostExchangeCert = bytes,
% HostSigningAuthority = uidref {AuthorityObjectUID},
% HostSigningCert = bytes,
% SessionTimeout = uinteger,
% TransTimeout = uinteger,
% InitialCredit = uinteger,
% SignedHash = bytes ]
% =>
% SMUID.SyncSession [ see SyncSession definition in 5.2.3.2]
% \end{lstlisting}

% Each method call is defined by the caller UID, method UID, and values of its parameters.
% The method calls are coded using tokens... start list, end list, optional argument start, ... short int, long int,....

% To call a method use IF-SEND command, to get result of the method call use IF-RECV command.

% In order to authentize the user, one can use many different ...



% Methods are sent using packets. There are three types of packets: ComPackets, packets and subpackets. A single ComPacket can contain multiple packets and single packet can contain multiple subpackets. Each ComPacket is associated with a single ComID, each packet is associated with a single session.



% \subsection{setup}
% \subsection{set locking range}

% global locking range is a special locking range that protects LBAs not protected by any of the other locking ranges.

% \subsection{access...}
% \subsection{...}


% \section{Comparison with OPAL 1.0}

% OPAL 2 disks are \emph{optionally} backwards compatible.

% cite \url{https://trustedcomputinggroup.org/wp-content/uploads/TCG_Storage-Opal_SSC_FAQ.pdf}

% \section{Comparision with other TCG's SSC}

% Enterprise, Opalite, Pyrite, Ruby


% hMM:

% In fact, the test of 4-byte ”SYN” RNG values has also been used to verify that a My Passport model with
% VID:PID 1058:0820, using a different JMicron chip (JMS569), in fact uses the same RNG. However
% this does not seem to be a big weakness for this model, since the JMS569 does not support HW AES
% and this model only supports ”AES mode” 0x30 (FDE). This mode does not accept any host provided
% key material when erased. All authentication and encryption is done in the HDD itself.

% How does that not make it "big weakness"? when the DEK can still be easily guessed?

% The only possible ”AES Mode” supported is 0x30, which
% refers to the FDE option. 

% What's this AES Mode?

% In case the DEVSLP signal is received, all
% secret key information present in SRAM is encrypted using a
% hardcoded key. The result is copied to DRAM. Subsequently,
% four ‘magic’ numbers are written to DRAM, and finally, the
% cores and SRAM are powered down

% SRAM switched with DRAM?
