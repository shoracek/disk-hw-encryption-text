
\chapter{Existing tools for Linux}
\label{chapter_tools}

In order to give users the ability to manage their SED disks without the need to create their own programs to access existing interfaces, there exist several tools implementing the protocols introduced in Chapter~\ref{chapter_hardware_disk_encryption} and Chapter~\ref{chapter_opal}. 

Note that some of the tools use the ATA TRUSTED commands for managing ATA Opal devices. When using these commands, it is important to perform the necessary setup, as described in Section~\ref{section:direct_communication_raw_ioctl}.

In this chapter, we will look at several existing popular open-source tools that allow one to manage the encryption of disks.

% it is necessary to first set the \verb|allow_tpm|\footnote{The usage of TPM in the name appears to be purely a historical blunder as it does not seem to be ever used for TPM, a cryptoprocessor standard defined by TCG, the same group defining TCG Storage standards.} flag of the kernel libATA library to allow Trusted Computing commands~\cite{acs-3}. This can be done either by specifying it as a kernel flag \verb|libata.allow_tpm=1|\footnote{\url{https://wiki.archlinux.org/title/Self-encrypting_drives}}.\todo[color=green]{where to find the instructions?, \url{https://wiki.archlinux.org/title/Self-encrypting_drives}}

% Some disks might provide their own software for their own proprietary hardware encryption. These are usually found separately on website or something, but some are provided as by the disk on an unecrypted partition.

% \section{Cryptsetup}

% \emph{Cryptsetup}\footnote{\url{https://gitlab.com/cryptsetup/cryptsetup}} is a tool for disk encryption setup. Although this tool supports several formats and volumes, such as ..., all of them are software, and there is no hardware encryption support as of now.\todo{probably remove this since it does not support any hardware encryption,,, yet}

\section{hdparm}
 
The \emph{hdparm}\footnote{\url{https://sourceforge.net/projects/hdparm}} tool provides a command line interface to control the parameters of ATA disks. Among others, this tool gives access to control ATA Security Feature Set.
The examples of SED-relevant commands for the functionality described in Subsection~\ref{subsection:enc_ata} are:
\begin{itemize}
\item Enable the ATA Security Feature Set:\\ \verb|hdparm --security-set-pass pwd /dev/sda|
\item Disable ATA Security Feature Set:\\ \verb|hdparm --security-disable pwd /dev/sda|
\item Unlock the disk:\\ \verb|hdparm --security-unlock pwd /dev/sda|
\item Erase the disk:\\ \verb|hdparm --security-erase pwd /dev/sda|
\end{itemize}
Implicitly the user password is referred to in the commands, but \verb|--user-master| can be specified to select master or user password explicitly. While enabling the ATA Security Feature Set, it is also possible to use the \verb|--security-mode| option to choose between high and maximum security modes.
A command to lock the drive is missing because a drive with an enabled ATA Security Feature Set is locked when powered off.

% ATA Security Feature Set: seems to be juts for ata and ata password, can it also work with our disks?

\section{sedutil}

\emph{sedutil}\footnote{\url{https://github.com/Drive-Trust-Alliance/sedutil}} is a tool for the management of SED disks maintained by the Drive Trust Alliance.
It currently supports the Enterprise and Opal SSCs (and additionally the remaining Pyrite, Opalite and Ruby SSCs in a fork of the project\footnote{\url{https://github.com/ChubbyAnt/sedutil}}) and NVMe and SATA interfaces.

Even though it is currently the most prominent open-source project to control SED disks, being exclusively used in guides for using SED on Linux, the code repository is inactive.



This tool does not use the Linux Opal ioctl interface and instead uses the approach described in Section~\ref{section:direct_communication_raw_ioctl}. This allows the tool not only to be multi-platform but also to use the Opal features in their entirety instead of only the subset introduced in the Opal ioctl interface.


The tool offers the following commands, grouped by their functionality:
\begin{itemize}
    \item Discovery: \begin{itemize}
\item \verb|isValidSED| command lists SSCs supported by the selected disk, the name of the disk and its firmware version.
\item \verb|scan| command performs \verb|isValidSED| command on every disk in the system.
\item \verb|query| command performs Level 0 and Level 1 Discovery on the selected disk.
\item \verb|printDefaultPassword| command prints the MSID password of the selected disk.
    \end{itemize}
    
    \item TPer management: \begin{itemize}
\item \verb|initialSetup| command takes ownership of the disk, initialises the Locking SP and the global LR, and enables MBR.
\item \verb|setSIDPassword| command changes the password of the SID authority.
\item \verb|setAdmin1Pwd| command changes the password of the Locking SP's first admin authority. 
\item \verb|setPassword| command changes the password of any supported Locking SP's authority.
    \end{itemize}
    
    \item Locking range management: \begin{itemize}
\item \verb|listLockingRange| command prints information about a single LR, such as the start, length or lock status.
\item \verb|listLockingRanges| command prints information about all LRs, similar to the previous command \verb|listLockingRange|.
\item \verb|rekeyLockingRange| command re-generates the selected LR's key, destroying the data in the LR in the process.
\item \verb|setupLockingRange| command sets the start and length of the locking range.
\item \verb|setLockingRange| command sets ReadLocked and WriteLock of the LR depending on the chosen configuration (read-write, read-only, or locked).
\item \verb|enableLockingRange| command sets ReadLockEnabled and WriteLockEnabled of the LR. 
\item \verb|disableLockingRange| command unsets ReadLockEnabled and WriteLockEnabled of the LR. 
    \end{itemize}
    
    \item Shadow MBR management: \begin{itemize}
\item \verb|setMBREnable| command sets Enable column of the MBRControl table.
\item \verb|setMBRDone| command sets the Done column of the MBRControl table.
\item \verb|loadPBAimage| command writes the contents of a file into the MBR table to be used as the shadow MBR.
    \end{itemize}

\begin{samepage}
    \item Reverting the device: \begin{itemize}
\item \verb|revertTPer| command reverts the TPer using the SID authority.
\item \verb|yesIreallywanttoERASEALLmydatausingthePSID| or the undocumented \verb|PSIDrevert| command reverts the TPer to factory state using the PSID password.
\item \verb|revertNoErase| command reverts only the Locking SP, keeping the global range data.
    \end{itemize}
\end{samepage}
    
    \item Enterprise-specific functionality\todo[color=green]{actually find out what the bands are first, seem like just locking ranges, but hard to find source, .... yea}: \begin{itemize}
\item \verb|setBandsEnabled| command enables the usage of all locking ranges.
\item \verb|setBandEnabled| command enables the usage of selected locking range.
\item \verb|eraseLockingRange| command cryptographically erases the data in a specified locking range.
    \end{itemize}
\end{itemize}




\todo[color=green]{exclusively as it is the only one mentioned, not that it is not mentioned elsewhere..., hmmm , sounds fine i guess}
% ... although there currently certain limits for e.g. having only one admin ...

% Provides functionality to:
% For Discovery there are commands 
% \verb|isValidSED| which lists supported SSCs by the selected disk,
% \verb|scan| which performs \verb|isValidSED| command on every disk in the system,
% and \verb|query| which performs Level 0 and Level 1 Discovery on the selected disk.
% Command \verb|printDefaultPassword| prints the MSID password.
% For management of the disk there are commands \verb|initialSetup| enabling to take ownership of the disk, 
% \verb|setSIDPassword| changing the password of SID authority,
% \verb|setAdmin1Pwd| changing password of Locking SP's first admin authority, 
% and \verb|setPassword| changing the password of any Locking SP's authority.

% For management of LRs there commands
% \verb|listLockingRanges| prints information about all LRs,
% \verb|listLockingRange| prints information about a single LR,
% \verb|rekeyLockingRange| regenerates the LR's key (destroying the data in the LR in the process)
% \verb|setupLockingRange| sets the start and length of locking range,
% \verb|setLockingRange| sets read locked and write locked of the LR, and
% \verb|disableLockingRange| and \verb|enableLockingRange| reading/writing locking enabled, 


% For controlling the shadow MBR there are commands \verb|setMBREnable| and \verb|setMBRDone| to set the Enable and Done columns respectively, and \verb|loadPBAimage| to write a file into the MBR table to be used as the shadow MBR.

% There are several commands for reverting the state of the device. Command
% \verb|revertTPer| reverts the TPer using the MSID password, 
% \verb|yesIreallywanttoERASEALLmydatausingthePSID| reverts the TPer using the PSID password, 
% and \verb|revertNoErase| reverts only the Locking SP, keeping the global range data.

% TCG Enterprise specific functionality, replaces LRs???? not really???, the use bands and bandmasters and erasemasters.
% \verb|setBandsEnabled|
% \verb|setBandEnabled|
% \verb|eraseLockingRange| 

\section{go-tcg-storage}


\emph{go-tcg-storage}\footnote{\url{https://github.com/open-source-firmware/go-tcg-storage}} is a Go library for managing TCG Storage disks, also providing CLI tools. Even though when compared to the previously introduced sedutil, it is not as popular, it is currently in active development. It supports not only Opal 2.0 and Enterprise SSCs but also Ruby, the successor of Enterprise. It also, compared to sedutil, supports, in addition, the SAS interface.
Although it implements a CLI, its capabilities compared to sedutil are very limited, as can be seen from the following paragraphs. The CLI consists of three utilities: \verb|sedlockctl|, \verb|tcgdiskstat|, and \verb|tcgsdiag|. 
For some of the tested disks, the utilities were not able to start a session due to time out and therefore function correctly.



The \verb|sedlockctl| utility allows one to control a SED disk. At the time of writing, it offers the following commands: \begin{itemize}
 \item \verb|list| command lists all LRs.
 \item \verb|lock-all| command locks all LRs.
 \item \verb|unlock-all| command unlocks all LRs.
 \item \verb|mbrdone| command sets the MBRDone cell to the desired value.
 \item \verb|read-mbr| command prints the shadow MBR.
\end{itemize}

The \verb|tcgdiskstat| utility provides information about supported SSCs and firmware version for each disk, similar to the sedutil's \verb|scan| command. Unlike sedutil's \verb|scan|, it also prints whether encryption is supported, MBR is enabled, or SID authority is blocked.

The \verb|tcgsdiag| utility performs Level 0 and Level 1 Discovery, similar to sedutil's \verb|query| command. It also prints the MSID password, random numbers generated by the disk, and contents of the TPerInfo table. 


% \todo{that does everything and is actually supported.}
\todo[color=green]{It has a few tools that let you do some things. It also supports more SSCs than others. Just found out about it, will need to look into it deeper and figure out why is it worse, so we won't look as bad.}
\todo[color=green]{CLI: Was able to generate random numbers, but only for some of the disks. and very limited control over it}

\section{TCGstorageAPI}

Seagate's \emph{TCGstorageAPI}\footnote{\url{https://github.com/Seagate/TCGstorageAPI}} is a library that provides a Python interface and a CLI tool. Similarly to sedutil, it supports Opal and Enterprise SSCs.
Compared to the previous tools, this one offers the most functionality, but this comes at the cost of also the largest dependencies. The tool has the most complete support for Enterprise SSC.
The library also includes a command to test an implementation of Opal drive, although it is not as complete as the test cases published by TCG.

When used with a device implementing Opal 1 SSC, the CLI tool reported ``SED configuration is Unknown/Unsupported (Type Opal) - Exiting Script'', even though the device also implemented a supported Opal 2 SSC.

\todo[color=green]{CLI: doesn't work with sdc "SED configuration is Unknown/Unsupported (Type Opal) - Exiting Script", no random numbres} 


% \todo{Provides a Python library, written in c++, and a CLI...}

% \todo{Has a bunch of dependencies, most importantly boost, which is pretty big...}


\todo[color=green]{offers test suite!!!}

\section{sedcli}

\emph{sedcli}\footnote{\url{https://github.com/sedcli/sedcli}} is another tool for managing SED devices. Although the only supported disk interface is NVMe and the only supported SSC is Opal, it aims to support the Key Management Interoperability Protocol and key management servers, allowing automatic provision and unlocking of Opal disks.

% However, the key management is not implemented yet, and out of all the tools 

\section{fscrypt}

Compared to the previously mentioned SED solutions, \emph{fscrypt} offers support for inline encryption instead. This allows it to be more flexible and provide filesystem-level encryption.
Since inline encryption, and especially filesystem-level encryption, require the active participation of the host during the encryption, support in the kernel is required in addition to the user space tool.
In the case of fscrypt, the kernel support is provided by a kernel library also called fscrypt.


\subsection{Kernel space library}

The fscrypt kernel library is currently the only kernel space library providing support for inline encryption, although, by default, it uses software encryption.
This library works on a filesystem level, which allows it to be more flexible compared to block device-level encryption, as it can limit the encryption to only some files. However, fscrypt does not encrypt all information of the files, and the metadata (excluding the filename) are left unencrypted.



\subsection{User space tool}

The user space tool fscrypt\footnote{\url{https://github.com/google/fscrypt}} allows one to manage Linux filesystem encryption. Although it does not explicitly support inline encryption, thanks to the kernel space library fscrypt providing inline encryption in a transparent way, it can still be used to manage file system encryption with inline encryption.


As of now, block layer inline encryption is supported only by two file systems in Linux: ext4 and F2FS. 
In the following paragraphs, we will focus only on the ext4 filesystem.

First of all, inline encryption needs to be enabled in the kernel configuration. This means that the Linux kernel configuration option \verb|CONFIG_FS_ENCRYPTION_INLINE_CRYPT| needs to be enabled.
In order to start using inline encryption on a file system, it needs to be mounted with the option to specify the usage of inline encryption.
\begin{lstlisting}[language=bash]
mount -t ext4 /dev/foo /mnt/foo -o inlinecrypt
\end{lstlisting}
It is not enough to only specify the inline encryption flag. The encryption itself also must be enabled in the file system. On ext4 file systems, the encryption can be enabled after mounting like so:
\begin{lstlisting}[language=bash]
tune2fs -O encrypt /dev/foo
\end{lstlisting}
After this, fscrypt can be used as usual, and it will use inline encryption for this filesystem.



% Compared to the device-level encryption it is able to limit the encryption to only some files.
% \emph{fscrypt} is the name of a user space tool and kernel library. Together, they offer a way to provide a filesystem level encryption
% This means that in order to provide inline encryption, it is not enough to have only a user space tool.
% This is why there is fscrypt the kernel library and fscrypt the user space tool.





% In the upstream Linux, there exists a support for inline encryption. However, it is currently used only by the kernel space library \emph{fscrypt}.
% Although by default fscrypt uses software encryption, it is possible to switch to inline encryption.

% The steps to activating inline encryption are described in more detail late in Chapter~\ref{chapter_tools}.
% There also exists a user space application with the same name\footnote{\url{https://github.com/google/fscrypt}}, that can be used to control encryption of file systems using kernel space fscrypt.


% Even though the \emph{fscrypt} tool does not provide a way 
% Since the kernel space library fscrypt provides inline encryption in a transparent way, it is possible to use it 
% The user space library \emph{fscrypt} offers a way to configure an inline encryption for a given file descriptor. No CLI is provided and so the C library must be used instead.



% Although it is possible to use the kernel space fscrypt library directly, there exist a user space utility with same name.

% In order to encrypt a folder using the kernel space fscrypt, the following must be done: an encryption key must be added and the encryption policy must be created. In C, this can be done the following way:
% \begin{lstlisting}[language=c]
% int fd = open(pathname, O_RDONLY | O_CLOEXEC);

% // add a key
% struct fscrypt_add_key_arg *key_request = calloc(1, sizeof(struct fscrypt_add_key_arg) + key_len);

% key_request->key_spec.type = FSCRYPT_KEY_SPEC_TYPE_IDENTIFIER;
% key_request->key_id = 0;
% key_request->raw_size = key_len;
% memcpy(key_request->raw, key, key_len);
% ioctl(fd, FS_IOC_ADD_ENCRYPTION_KEY, key_request);

% // set a policy
% struct fscrypt_policy_v2 policy_request = { 0 };

% policy_request.version = FSCRYPT_POLICY_V2;
% policy_request.contents_encryption_mode = FSCRYPT_MODE_AES_256_XTS;
% policy_request.filenames_encryption_mode = FSCRYPT_MODE_AES_256_CTS;
% policy_request.flags = FSCRYPT_POLICY_FLAGS_PAD_8 | FSCRYPT_POLICY_FLAGS_PAD_16 | FSCRYPT_POLICY_FLAGS_PAD_32;
% memcpy(policy_request.master_key_identifier, key_request->key_spec.u.identifier, FSCRYPT_KEY_IDENTIFIER_SIZE);
% ioctl(fd, FS_IOC_SET_ENCRYPTION_POLICY, &policy_request);
% \end{lstlisting}
% This code sets up an encryption policy for the file specified by the pathname. 



\todo[color=green]{probably rework into text about the tool for management instead?.......}
\todo[color=green]{requires the filesystem to actually use the library In the following paragraphs, that should be redone, I am assuming ext4......}

% \subsection{Usage}


% \subsection{Implementation}
% \todo{maybe remove completely?, doesn't seem to fit much...}

% During mounting the ``inlinecrypt''/\verb|SB_INLINECRYPT|  flag is written into the \verb|super_block| structure.

% It all starts in \verb|__ext4_new_inode|. This is the internal function used when creating new inodes, called by functions such as \verb|ext4_create| when creating a new file.

% The function (if it is not inode used for large extended attributes?) calls \verb|fscrypt_prepare_new_inode|.

% \verb|fscrypt_prepare_new_inode -> fscrypt_setup_encryption_info ->| \verb|setup_file_encryption_key ->  fscrypt_select_encryption_impl|

% In \verb|fscrypt_select_encryption_impl| there is actually the only place where the \verb|SB_INLINECRYPT| flag is used.
% ... Calls \verb|blk_crypto_config_supported| to check the device's crypto profile.
% Afterwards, \verb|fscrypt_select_encryption_impl| function sets the \verb|(fscrypt_info *)ci->ci_inlinecrypt|.

% \verb|setup_per_mode_enc_key| then sets the \verb|(fscrypt_info *)ci->ci_enc_key|.

% \paragraph{fscrypt - How does it work? - Usage}

% The bio function is stored in \\ \verb|(struct bio *)bio->(struct bio_crypt_ctx *)bi_crypt_context|

% Function \verb|fscrypt_set_bio_crypt_ctx| changes the file's bio to use inline encryption... simply calls the blk layer \verb|bio_crypt_set_ctx|.

% Calling \verb|submit_bio| like normally ... \verb|__submit_bio| calls \verb|__blk_crypto_bio_prep|...

% \blockquote{If the bio crypt context provided for the bio is supported by the underlying device's inline encryption hardware, do nothing.}

% \verb|__blk_crypto_rq_bio_prep| however sets the context of the request to the one of the bio... 
% After the bio prep  \verb|blk_mq_submit_bio| gets called (which calls \verb|blk_mq_bio_to_request|, and after that also \verb|blk_crypto_init_request->blk_crypto_get_keyslot| which updates the devices keyslot to contain the new key..., but does nothing if the device does not have keyslots)..... the info about the key to use then has to be acquired by the driver from \verb|request->|


% Most important are probably structures \verb|blk_crypto_ll_ops| and \verb|blk_crypto_profile|... just two operations, program key and evict key. 



% how to get crypto profile from outside..




% Where does the hardware come to play?

% \verb|ufshcd_exec_raw_upiu_cmd()->ufshcd_issue_devman_upiu_cmd()->ufshcd_prepare_req_desc_hdr()->ufshcd_prepare_req_desc_hdr_crypto()| sets the header with the correct keyslot.

